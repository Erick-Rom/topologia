\documentclass[12pt]{article}
\usepackage[spanish]{babel}
\usepackage{amssymb}
\usepackage{amsmath}
\setlength{\parindent}{0pt}
%Falta agregar una portada bien hecha aparte.
\LARGE{\title{Tareas de segundo parcial-Topología}}
\author{Alumnos: \\Arturo Rodriguez Contreras - 2132880 \\
Jonathan Raymundo Torres Cardenas - 1949731\\
Praxedis Jimenes Ruvalcaba \\
Erick Román Montemayor Treviño - 1957959 \\
Alexis Noe Mora Leyva \\
Everardo Flores Rivera - 2127301}
\begin{document}
\maketitle

\paragraph{1}
\textit{Sea $f,g:X\rightarrow$ Y funciones continuas y Y bajo la top. del orden. Sea $h(x)=\min{f(x),g(x)}$. Demostrar que h es continua en $X$}
%Demostracion

\paragraph{2}
\textit{Sea $f:X\leftarrow Y$ una función abierta. Si $S \subset Y$ y $C$ cerrado en X tal que $f^{-1}(S)\subset C$, entonces existe K cerrado en Y tal que S$\subset$K y $f^{-1}(K)\subset C$}
%Demostracion

\paragraph{3}
\textit{Caso 1 y Caso 2 de ejemplo clase del 12/03/23} 


\paragraph{4}
\textit{Ver que $h^{-1}=g$ es continua en [a,b]}

\paragraph{5}
\textit{Demostrar que la relación entre esp. top. $X\sim Y$ es de equivalencia}

\paragraph{6}
\textit{Demostrar que $X\times Y \approx Y \times X$, extenderlo a caso finito utilizando cualquier permutación.}

\paragraph{7}
\textit{Verificar que si $A_{\alpha}\subset X_{\alpha}$, entonces $\prod\limits_{\alpha\in J } int(A_{\alpha}) = int(\prod\limits_{\alpha\in J }A_{\alpha})$ en la topologia por cajas.}

\paragraph{8}
\textit{Verificar si las $\beta$-esima proyecciones son abiertas y/o cerradas en top.}

\paragraph{9}

\end{document}