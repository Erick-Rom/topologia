\documentclass[12pt]{article}
\usepackage[spanish]{babel}
\usepackage{amssymb}
\usepackage{amsmath}
\usepackage{geometry}
 \geometry{
 letterpaper,
 total={170mm,237mm},
 left=20mm,
 top=15mm,
 }
\usepackage{setspace}
\spacing{1.5}
\setlength{\parindent}{0pt}
%Falta agregar una portada bien hecha aparte.
\LARGE{\title{Tareas de segundo parcial-Topología}}
\author{Alumnos: \\Arturo Rodriguez Contreras - 2132880 \\
Jonathan Raymundo Torres Cardenas - 1949731\\
Praxedis Jimenes Ruvalcaba \\
Erick Román Montemayor Treviño - 1957959 \\
Alexis Noe Mora Leyva \\
Everardo Flores Rivera - 2127301}
\begin{document}
\maketitle

\paragraph{1}
\textit{Sea $f,g:X\rightarrow$ Y funciones continuas y Y bajo la top. del orden. Sea $h(x)=\min{f(x),g(x)}$. Demostrar que h es continua en $X$}
%Demostracion

\paragraph{2}
\textit{Sea $f:X\leftarrow Y$ una función abierta. Si $S \subset Y$ y $C$ cerrado en X tal que $f^{-1}(S)\subset C$, entonces existe K cerrado en Y tal que S$\subset$K y $f^{-1}(K)\subset C$}
%Demostracion

\paragraph{3}
\textit{Caso 1 y Caso 2 de ejemplo clase del 12/03/23} 


\paragraph{4}
\textit{Ver que $h^{-1}=g$ es continua en [a,b]}

\paragraph{5}
\textit{Demostrar que la relación entre esp. top. $X\sim Y$ es de equivalencia}

\paragraph{6}
\textit{Demostrar que si $f(\overline{A})=\overline{f(A)}$ para cada $A\subset X$ entonces f es un homeomorfismo}

\paragraph{7}
\textit{Demostrar que $X\times Y \approx Y \times X$, extenderlo a caso finito utilizando cualquier permutación.}
 
\paragraph{8}
\textit{Demostrar que $\tau = \{\prod\limits_{\alpha\in J}U_\alpha : U_\alpha\in\tau_\alpha\}$ es una topologia para el producto y se le conoce como la topolofia por cajas}

\paragraph{9}
\textit{Verificar que si $A_{\alpha}\subset X_{\alpha}$, entonces $\prod\limits_{\alpha\in J } int(A_{\alpha}) = int(\prod\limits_{\alpha\in J }A_{\alpha})$ en la topologia por cajas.}

El resultado es en general falso. Tomemos $\mathbb{R}^\omega$, $A_n=\left(-1/n,1/n\right)$. Es facil ver que $\prod \text{int}(A_n)-\prod A_n$, pero si $U=\prod_{i=1}^{m} U_{n_i} \times \prod_{n\neq n_i} \mathbb{R} \subset \prod A_n$,
entonces $x_{n_i} \in U_{n_i} \cap A_{n_i}$ y $x_n=1$ si $n \neq n_i$, cumple que $(x_n) \in U$, pero $(x_n) \notin \prod A_n$.
\paragraph{10}
\textit{Verificar si las $\beta$-esima proyecciones son abiertas y/o cerradas en ambas topologias}

Sea $U=\prod_{\alpha \in J} U_\alpha$ y note que $\pi_\beta(U)=U_\beta$, por lo que si $U$ es abierto en la top. por cajas o producto, en ambos casos $\pi_\beta(U)$ es abierto en $X_\beta$.
Ademas, de la igualdad $\overline{U}=\prod_{\alpha \in J} \overline{U_\alpha}$, que se cumple en ambas topologias, se sigue que $\pi_\beta(U)=\overline{U_\beta}$ es cerrado, es decir, $\pi_\beta$ es un mapeo abierto y cerrado.
\paragraph{11}
\textit{Sea $f:X\rightarrow Y$ con la topologiamétrica en $X\times Y$. Demostrar que f es continua en X si y solo si $\forall\epsilon>0\hspace{5px}\exists\delta>0 : f(B_{d_x}(x,\delta))\subset B_{d_y}(f(x),\epsilon) \forall x\in X$}

\paragraph{12}
\textit{Demostrar que la métrica uniforme $\rho$ es métrica.}

Por definición, $\rho((x_n),(y_n))=\sup{\{\overline{d}(x_n,y_n)\}}$, donde $\overline{d}(x,y) \leq 1$ para cada $x,y \in \mathbb{R}$, por lo que $\rho$ está bien definida.
Es claro que $\rho((x_n),(x_n))=0$, además, $(x_n) \neq (y_n)$ implica que existe un natural $m$ con $\rho((x_n),(y_n))>=\overline{d}(x_m,y_m)>0$. Por tanto, $\rho((x_n),(y_n))=0$, si y sólo si $(x_n)=(y_n)$.
La simetria se hereda de la metrica acotada, $\rho((x_n),(y_n))=\sup{\{\overline{d}(x_n,y_n)\}}=\sup{\{\overline{d}(y_n,x_n)\}}=\rho((y_n),(x_n))$. Finalmente, veamos la desigualdad triangular.
\begin{align*}
    \rho((x_n),(y_n)) &\leq \sup{\{\overline{d}(x_n,z_n)+\overline{d}(z_n,y_n)\}} \\
                      &\leq \sup{\{\overline{d}(x_n,z_n)\}}+\sup{\{\overline{d}(z_n,y_n)\}}\\
                      &=\rho((x_n),(z_n))+\rho((z_n),(y_n)).
\end{align*}
\paragraph{13}
\textit{Sea $A=\{(x_n)_{n\in\mathbb{N}}\in\mathbb{R}^\omega : \exists N\in\mathbb{N}: x_n =0 ; n \geq N \}$
Hallar $\overline{A}$ en top. uniforme}

\paragraph{14}
\textit{Sea A del ejercicio anterior, hallar $\overline{A}$ en top. cajas}

\paragraph{15}
\textit{Demostrar que $f^{-1}(Fr_Y(B))\subset Fr_X(f^{-1}(B))$}

\paragraph{16}
\textit{Sea $h:\mathbb{R}^\omega\rightarrow\mathbb{R}^\omega$ definida por: $h((x_n)_{n\in\mathbb{N}})=(a_n x_n +b_n)_{n\in\mathbb{N}}$. Ver si h es homeomorfismo en $\mathbb{R}^\omega$ bajo top. cajas}

Si $a_m=0$ para algun $m \in \mathbb{N}$, dado $(x_n)$, definimos $(y_n)$ por $y_n=x_n$ si $n \neq m$, $y_m=x_m+1$. Es claro que $(x_n) \neq (y_n)$, pero $h((x_n))=h((y_n))$, por lo que h no es biyectiva y por tanto, no puede ser homeomorfismo.
Supongamos entonces que $a_n \neq 0$ para cada $n \in \mathbb{N}$. Observe que $h^{-1}((x_n))=(\frac{x_n-b_n}{a_n})$ es la función inversa de $h$. Tanto $h$ como $h^{-1}$ son de la forma $f((x_n))=(c_n x_n+d_n)$, por lo que basta probar que esta función
es continua en la topología por cajas. Sea $p_n((x_n))=c_nx+d_n$, es facil ver que es continua para cada $n \in \mathbb{N}$, y sea $U=\prod_{i=1}^{m} U_{n_i} \times \prod_{n\neq n_i} \mathbb{R}$ un abierto en $\mathbb{R}^\omega$,  $f^{-1}(U)=\cap_{i=1}^{m} p^{-1}_{n_i}(U_{n_i})$ es una
intersección finita de abiertos, por lo que es abierta y $f$ es continua. Por tanto, $h$ es un homeomorfismo.
\end{document}