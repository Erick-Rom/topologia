\documentclass[12pt]{article}
\usepackage[spanish]{babel}
\usepackage{amssymb}
\usepackage{amsmath}
\usepackage{geometry}
 \geometry{
 letterpaper,
 total={170mm,237mm},
 left=20mm,
 top=15mm,
 }
\usepackage{setspace}
\spacing{1.5}
\setlength{\parindent}{0pt}
%Falta agregar una portada bien hecha aparte.
\LARGE{\title{Tareas de segundo parcial-Topología}}
\author{Alumnos: \\Arturo Rodriguez Contreras - 2132880 \\
Jonathan Raymundo Torres Cardenas - 1949731\\
Praxedis Jimenes Ruvalcaba \\
Erick Román Montemayor Treviño - 1957959 \\
Alexis Noe Mora Leyva \\
Everardo Flores Rivera - 2127301}
\begin{document}
\maketitle

\paragraph{1}
\textit{Sea $f,g:X\rightarrow$ Y funciones continuas y Y bajo la top. del orden. Sea $h(x)=\min{f(x),g(x)}$. Demostrar que h es continua en $X$}
%Demostracion

\paragraph{2}
\textit{Sea $f:X\leftarrow Y$ una función abierta. Si $S \subset Y$ y $C$ cerrado en X tal que $f^{-1}(S)\subset C$, entonces existe K cerrado en Y tal que S$\subset$K y $f^{-1}(K)\subset C$}
%Demostracion

\paragraph{3}
\textit{Caso 1 y Caso 2 de ejemplo clase del 12/03/23} 


\paragraph{4}
\textit{Ver que $h^{-1}=g$ es continua en [a,b]}

\paragraph{5}
\textit{Demostrar que la relación entre esp. top. $X\sim Y$ es de equivalencia}

\paragraph{6}
\textit{Demostrar que si $f(\overline{A})=\overline{f(A)}$ para cada $A\subset X$ entonces f es un homeomorfismo}

Supongamos que \( f : X \to Y \) es una función tal que para todo subconjunto \( A \subset X \), se cumple que
\[
f(\overline{A}) = \overline{f(A)}.
\]

P. D. \( f \) es continua: Sea \( C \subset Y \) cerrado. Consideramos \( A = f^{-1}(C) \subset X \). Por demostrar que \( A \) es cerrado
Por hipótesis
\[
f(\overline{A}) = \overline{f(A)} = \overline{f(f^{-1}(C))} \subset \overline{C} = C.
\]
entonces,
\[
\overline{A} \subset f^{-1}(f(\overline{A})) \subset f^{-1}(C) = A.
\]
por lo tanto, \( A \) es cerrado y \( f \) es continua

P.D. \( f \) es inyectiva: Supongamos que \( f(x_1) = f(x_2) \). Tomamos \( A = \{x_1\} \) y notamos que
\[
f(\overline{A}) = \overline{f(\{x_1\})} = \overline{\{f(x_1)\}} = \overline{\{f(x_2)\}} = \overline{f(\{x_2\})} = f(\overline{\{x_2\}}).
\]
Entonces \( \overline{\{x_1\}} \) y \( \overline{\{x_2\}} \) tienen la misma imagen bajo \( f \). Así que, \( f \) debe ser inyectiva.

P.D. \( f \) es sobreyectiva 
Sea \( y \in Y \). Supongamos que no existe \( x \in X \) tal que \( f(x) = y \). Entonces \( y \notin f(X) \), y por tanto \( y \notin \overline{f(A)} \) para ningún \( A \subset X \).  
Pero si \( y \notin \overline{f(A)} \) para todo \( A \), entonces no existe \( A \subset X \) tal que \( y \in f(\overline{A}) \), pero esto contradice que la cerradura de \( A \) debe contener todos los puntos límite de las imágenes.  
Por lo tanto, \( y \in f(X) \), y \( f \) es sobreyectiva.


P.D: \( f^{-1} \) es continua: Como \( f \) es biyectiva, para todo \( B \subset Y \) existe \( A \subset X \) tal que \( B = f(A) \). Entonces:
\[
f(\overline{A}) = \overline{f(A)} = \overline{B}.
\]
Aplicando \( f^{-1} \),
\[
f^{-1}(\overline{B}) = f^{-1}(f(\overline{A})) = \overline{A} = \overline{f^{-1}(B)}.
\]
Así, \( f^{-1}(\overline{B}) = \overline{f^{-1}(B)} \), lo que implica que \( f^{-1} \) también es continua (por el mismo argumento del paso 1).

Como \( f \) es continua, biyectiva y su inversa también es continua. Por lo tanto, \( f \) es un homeomorfismo. 


\paragraph{7}
\textit{Demostrar que $X\times Y \sim Y \times X$, extenderlo a caso finito utilizando cualquier permutación.}

Sea $f: X \times Y \to Y \times X$ definida por $f(x,y)=(y,x)$. Es claro que $f$ es biyectiva, veamos que es un homeomorfismo. Sea $A=V \times U$ un basico de $Y \times X$, 
tenemo que $f^{-1}(V \times U)=U \times V$ es abierto en $X \times Y$. Con un argumento similar obtenemos que $f^{-1}$ es continua, por tanto $f$ es un homeomorfismo.

Ahora demostraremos que si $\sigma \in S_n$ es una permutación, entonces $\prod_{k=1}^{n} X_k \sim \prod_{k=1}^{n} X_{\sigma(k)}$.
Recordemos que toda permutación se puede escribir como una composición de transposiciones de la forma $\pi(k)=k$ si $k<i$ o $k>i+1$ y $\pi(i)=i+1$, $\pi(i+1)=i$. Por lo primero demostrado,
y del hecho que la relación de homeomorfismo es de equivalencia, tenemos que $\sigma=\pi_1 \circ ...\circ \pi_m$, $$\prod_{k=1}^{n} X_k \sim \prod_{k=1}^{n} X_{\pi_1(k)} \sim \prod_{k=1}^{n} X_{(\pi_1 \circ \pi_2)(k)} \sim ... \sim \prod_{k=1}^{n} X_{\sigma(k)}$$

\paragraph{8}
\textit{Demostrar que $\mathcal{B} = \{\prod\limits_{\alpha\in J}U_\alpha : U_\alpha\in\tau_\alpha\}$ es una base para la topología del producto y se le conoce como la topología por cajas.}

Sean $A=\prod_{\alpha \in J}U_\alpha, B=\prod_{\alpha \in J} V_\alpha$. Como $A \cap B=\prod_{\alpha \in J} U_\alpha \cap V_\alpha = C \in \mathcal{B}$, entonces $\mathcal{B}$ es base.

\paragraph{9}
\textit{Verificar que si $A_{\alpha}\subset X_{\alpha}$, entonces $\prod\limits_{\alpha\in J } int(A_{\alpha}) = int(\prod\limits_{\alpha\in J }A_{\alpha})$ en la topologia por cajas.}

El resultado es en general falso. Tomemos $\mathbb{R}^\omega$, $A_n=\left(-1/n,1/n\right)$. Es facil ver que $\prod \text{int}(A_n)-\prod A_n$, pero si $U=\prod_{i=1}^{m} U_{n_i} \times \prod_{n\neq n_i} \mathbb{R} \subset \prod A_n$,
entonces $x_{n_i} \in U_{n_i} \cap A_{n_i}$ y $x_n=1$ si $n \neq n_i$, cumple que $(x_n) \in U$, pero $(x_n) \notin \prod A_n$.
\paragraph{10}
\textit{Verificar si las $\beta$-esima proyecciones son abiertas y/o cerradas en ambas topologias}

Sea $U=\prod_{\alpha \in J} U_\alpha$ y note que $\pi_\beta(U)=U_\beta$, por lo que si $U$ es abierto en la top. por cajas o producto, en ambos casos $\pi_\beta(U)$ es abierto en $X_\beta$.
Ademas, de la igualdad $\overline{U}=\prod_{\alpha \in J} \overline{U_\alpha}$, que se cumple en ambas topologias, se sigue que $\pi_\beta(U)=\overline{U_\beta}$ es cerrado, es decir, $\pi_\beta$ es un mapeo abierto y cerrado.
\paragraph{11}
\textit{Sea $f:X \to Y$, $X,Y$ espacios metricos. Demostrar que f es continua en X si y solo si $\forall\epsilon>0\hspace{5px}\exists\delta>0 : f(B_{d_x}(x,\delta))\subset B_{d_y}(f(x),\epsilon) \forall x\in X$}

Del teorema de equivalencia para continuidad, $f$ es continua si, y sólo si para cada basico $V=B_{d_Y}(f(x),\epsilon)$ existe un basico $U=B_{d_X}(x,\delta)$ tal que $f(U) \subset V$, esto es, $f(B_{d_X}(x,\delta)) \subset B_{d_Y}(f(x),\epsilon)$.

\paragraph{12}
\textit{Demostrar que la métrica uniforme $\rho$ es métrica.}

Por definición, $\rho((x_n),(y_n))=\sup{\{\overline{d}(x_n,y_n)\}}$, donde $\overline{d}(x,y) \leq 1$ para cada $x,y \in \mathbb{R}$, por lo que $\rho$ está bien definida.
Es claro que $\rho((x_n),(x_n))=0$, además, $(x_n) \neq (y_n)$ implica que existe un natural $m$ con $\rho((x_n),(y_n))>=\overline{d}(x_m,y_m)>0$. Por tanto, $\rho((x_n),(y_n))=0$, si y sólo si $(x_n)=(y_n)$.
La simetria se hereda de la metrica acotada, $\rho((x_n),(y_n))=\sup{\{\overline{d}(x_n,y_n)\}}=\sup{\{\overline{d}(y_n,x_n)\}}=\rho((y_n),(x_n))$. Finalmente, veamos la desigualdad triangular.
\begin{align*}
    \rho((x_n),(y_n)) &\leq \sup{\{\overline{d}(x_n,z_n)+\overline{d}(z_n,y_n)\}} \\
                      &\leq \sup{\{\overline{d}(x_n,z_n)\}}+\sup{\{\overline{d}(z_n,y_n)\}}\\
                      &=\rho((x_n),(z_n))+\rho((z_n),(y_n)).
\end{align*}
\paragraph{13}
\textit{Sea $A=\{(x_n)_{n\in\mathbb{N}}\in\mathbb{R}^\omega : \exists N\in\mathbb{N}: x_n =0 ; n \geq N \}$
Hallar $\overline{A}$ en top. uniforme}

\paragraph{14}
\textit{Sea A del ejercicio anterior, hallar $\overline{A}$ en top. cajas}

Veamos que $A=\overline{A}$. Sea $(x_n) \notin A$, luego existe una sucesión extrictamente creciente de naturales $(n_k)$ tal que $x_{n_k} \neq 0$ para cada $k \in \mathbb{N}$. Como $\mathbb{R}$ es $T_1$, existen vecindades abiertas $U_{n_k}$ de $x_{n_k}$ que no contienen al $0$.
Tomemos $U=\prod_{k=1}^{\infty}U_{n_k} \times \prod_{n \neq n_k} \mathbb{R}$. Es facil ver que $(x_n) \in U$. Ahora sea $(y_n) \in A \cap U$, por definición existe $N$ tal que $n>N$ entonces $y_n=0$, pero al ser $n_k \to \infty$, para algun $k$ se cumple que
$n_k>N$ y $y_{n_k} \in U_{n_k}$, contradiciendo el hecho que $0 \notin U_{n_k}$. Por tanto $A \cap U = \emptyset$ y $A=\overline{A}$.
\paragraph{15}
\textit{Demostrar que $f^{-1}(Fr_Y(B))\subset Fr_X(f^{-1}(B))$}

\paragraph{16}
\textit{Sea $h:\mathbb{R}^\omega\rightarrow\mathbb{R}^\omega$ definida por: $h((x_n)_{n\in\mathbb{N}})=(a_n x_n +b_n)_{n\in\mathbb{N}}$. Ver si h es homeomorfismo en $\mathbb{R}^\omega$ bajo top. cajas}

Si $a_m=0$ para algun $m \in \mathbb{N}$, dado $(x_n)$, definimos $(y_n)$ por $y_n=x_n$ si $n \neq m$, $y_m=x_m+1$. Es claro que $(x_n) \neq (y_n)$, pero $h((x_n))=h((y_n))$, por lo que h no es biyectiva y por tanto, no puede ser homeomorfismo.
Supongamos entonces que $a_n \neq 0$ para cada $n \in \mathbb{N}$. Observe que $h^{-1}((x_n))=(\frac{x_n-b_n}{a_n})$ es la función inversa de $h$. Tanto $h$ como $h^{-1}$ son de la forma $f((x_n))=(c_n x_n+d_n)$, por lo que basta probar que esta función
es continua en la topología por cajas. Sea $p_n((x_n))=c_nx+d_n$, es facil ver que es continua para cada $n \in \mathbb{N}$, y sea $U=\prod_{i=1}^{m} U_{n_i} \times \prod_{n\neq n_i} \mathbb{R}$ un abierto en $\mathbb{R}^\omega$,  $f^{-1}(U)=\cap_{i=1}^{m} p^{-1}_{n_i}(U_{n_i})$ es una
intersección finita de abiertos, por lo que es abierta y $f$ es continua. Por tanto, $h$ es un homeomorfismo.
\end{document}
