\documentclass[12pt]{article}
\usepackage[spanish]{babel}
\usepackage{amssymb}
\usepackage{amsmath}
\usepackage{geometry}
 \geometry{
 letterpaper,
 total={170mm,237mm},
 left=20mm,
 top=15mm,
 }
\usepackage{setspace}
\spacing{1.5}
\setlength{\parindent}{0pt}
%Falta agregar una portada bien hecha aparte.
\LARGE{\title{Tareas de segundo parcial-Topología}}
\author{Alumnos: \\Arturo Rodriguez Contreras - 2132880 \\
Jonathan Raymundo Torres Cardenas - 1949731\\
Praxedis Jimenes Ruvalcaba \\
Erick Román Montemayor Treviño - 1957959 \\
Alexis Noe Mora Leyva \\
Everardo Flores Rivera - 2127301}
\begin{document}
\maketitle

\paragraph{1}
\textit{Sea $f,g:X \to Y$ funciones continuas y Y bajo la top. del orden. Sea $h(x)=\min{\{f(x),g(x)\}}$. Demostrar que h es continua en $X$}

La topología del orden en \( Y \) tiene como base los conjuntos abiertos de la forma:
\[
(a, b),\quad (-\infty, b),\quad (a, \infty).
\]
Queremos probar que \( h \) es continua.  
Sea \( a \in Y \), consideremos un abierto básico \( U = (a, \infty) \subset Y \).  
Queremos ver que \( h^{-1}(U) \subset X \) es abierto.

observamos que
La función \( h(x) = \min\{f(x), g(x)\} \) puede reescribirse como:
\[
h(x) > a \iff \min\{f(x), g(x)\} > a \iff f(x) > a \text{ y } g(x) > a.
\]
Es decir:
\[
h^{-1}((a, \infty)) = f^{-1}((a, \infty)) \cap g^{-1}((a, \infty)).
\]
Como \( f \) y \( g \) son continuas y \( (a, \infty) \) es abierto en \( Y \), las preimágenes \( f^{-1}((a, \infty)) \) y \( g^{-1}((a, \infty)) \) son abiertas en \( X \).  
La intersección de abiertos es abierta, por lo tanto:
\[
h^{-1}((a, \infty)) \text{ es abierto en } X.
\]

Un razonamiento análogo muestra que:
\[
h^{-1}((-\infty, b)) = f^{-1}((-\infty, b)) \cup g^{-1}((-\infty, b)),
\]
lo cual también es abierto, ya que la unión de abiertos es abierta, y para los casos  (a,b) se pueden omitir pues es una combinación de intersecciones y uniones de los casos anteriores    
Dado que la preimagen por \( h \) de cualquier conjunto abierto básico de \( Y \) es abierta en \( X \), se concluye que \( h \) es continua.

\paragraph{2}
\textit{Sea $f:X\leftarrow Y$ una función abierta. Si $S \subset Y$ y $C$ cerrado en X tal que $f^{-1}(S)\subset C$, entonces existe K cerrado en Y tal que S$\subset$K y $f^{-1}(K)\subset C$}
%Demostracion

\paragraph{3}
\textit{Caso 1 y Caso 2 de ejemplo clase del 12/03/23} 


\paragraph{4}
\textit{Ver que $h^{-1}=g$ es continua en [a,b]}

\paragraph{5}
\textit{Demostrar que la relación entre esp. top. $X\sim Y$ es de equivalencia}

\paragraph{6}
\textit{Demostrar que si $f(\overline{A})=\overline{f(A)}$ para cada $A\subset X$ y ademas es biyectiva entonces f es un homeomorfismo}

Por teorema, basta probar que \( f \) es continua y cerrada.

P.D. \( f \) es cerrada.

Sea \( F \subset X \) cerrado. Entonces \( \overline{F} = F \). Aplicamos la hipótesis:
\[
f(F) = f(\overline{F}) = \overline{f(F)}.
\]
Esto implica que \( f(F) \) es cerrado en \( Y \), pues es igual a su clausura.  
Así, la imagen por \( f \) de cualquier cerrado en \( X \) es cerrado en \( Y \),  
lo que implica que \( f \) es cerrada.

Por último, note que de la hipótesis se sigue que $f(\overline{A}) \subset \overline{f(A)}$ y del teorema de equivalencia de continuidad, $f$ es continua.

\paragraph{7}
\textit{Demostrar que $X\times Y \sim Y \times X$, extenderlo a caso finito utilizando cualquier permutación.}

Sea $f: X \times Y \to Y \times X$ definida por $f(x,y)=(y,x)$. Es claro que $f$ es biyectiva, veamos que es un homeomorfismo. Sea $A=V \times U$ un basico de $Y \times X$, 
tenemo que $f^{-1}(V \times U)=U \times V$ es abierto en $X \times Y$. Con un argumento similar obtenemos que $f^{-1}$ es continua, por tanto $f$ es un homeomorfismo.

Ahora demostraremos que si $\sigma \in S_n$ es una permutación, entonces $\prod_{k=1}^{n} X_k \sim \prod_{k=1}^{n} X_{\sigma(k)}$.
Recordemos que toda permutación se puede escribir como una composición de transposiciones de la forma $\pi(k)=k$ si $k<i$ o $k>i+1$ y $\pi(i)=i+1$, $\pi(i+1)=i$. Por lo primero demostrado,
y del hecho que la relación de homeomorfismo es de equivalencia, tenemos que $\sigma=\pi_1 \circ ...\circ \pi_m$, $$\prod_{k=1}^{n} X_k \sim \prod_{k=1}^{n} X_{\pi_1(k)} \sim \prod_{k=1}^{n} X_{(\pi_1 \circ \pi_2)(k)} \sim ... \sim \prod_{k=1}^{n} X_{\sigma(k)}$$

\paragraph{8}
\textit{Demostrar que $\mathcal{B} = \{\prod\limits_{\alpha\in J}U_\alpha : U_\alpha\in\tau_\alpha\}$ es una base para la topología del producto y se le conoce como la topología por cajas.}

Sean $A=\prod_{\alpha \in J}U_\alpha, B=\prod_{\alpha \in J} V_\alpha$. Como $A \cap B=\prod_{\alpha \in J} U_\alpha \cap V_\alpha = C \in \mathcal{B}$, entonces $\mathcal{B}$ es base.

\paragraph{9}
\textit{Verificar que si $A_{\alpha}\subset X_{\alpha}$, entonces $\prod\limits_{\alpha\in J } int(A_{\alpha}) = int(\prod\limits_{\alpha\in J }A_{\alpha})$ en la topologia por cajas.}

El resultado es en general falso. Tomemos $\mathbb{R}^\omega$, $A_n=\left(-1/n,1/n\right)$. Es facil ver que $\prod \text{int}(A_n)-\prod A_n$, pero si $U=\prod_{i=1}^{m} U_{n_i} \times \prod_{n\neq n_i} \mathbb{R} \subset \prod A_n$,
entonces $x_{n_i} \in U_{n_i} \cap A_{n_i}$ y $x_n=1$ si $n \neq n_i$, cumple que $(x_n) \in U$, pero $(x_n) \notin \prod A_n$.
\paragraph{10}
\textit{Verificar si las $\beta$-esima proyecciones son abiertas y/o cerradas en ambas topologias}

Sea $U=\prod_{\alpha \in J} U_\alpha$ y note que $\pi_\beta(U)=U_\beta$, por lo que si $U$ es abierto en la top. por cajas o producto, en ambos casos $\pi_\beta(U)$ es abierto en $X_\beta$.
Ademas, de la igualdad $\overline{U}=\prod_{\alpha \in J} \overline{U_\alpha}$, que se cumple en ambas topologias, se sigue que $\pi_\beta(U)=\overline{U_\beta}$ es cerrado, es decir, $\pi_\beta$ es un mapeo abierto y cerrado.
\paragraph{11}
\textit{Sea $f:X \to Y$, $X,Y$ espacios metricos. Demostrar que f es continua en X si y solo si $\forall\epsilon>0\hspace{5px}\exists\delta>0 : f(B_{d_x}(x,\delta))\subset B_{d_y}(f(x),\epsilon) \forall x\in X$}

Del teorema de equivalencia para continuidad, $f$ es continua si, y sólo si para cada basico $V=B_{d_Y}(f(x),\epsilon)$ existe un basico $U=B_{d_X}(x,\delta)$ tal que $f(U) \subset V$, esto es, $f(B_{d_X}(x,\delta)) \subset B_{d_Y}(f(x),\epsilon)$.

\vspace{1em}
Supongamos que \( f \) es continua
Sea \( x \in X \) y \( \epsilon > 0 \). Entonces el conjunto \( B_{d_Y}(f(x), \epsilon) \subset Y \) es abierto.  
Como \( f \) es continua, la preimagen \( f^{-1}(B_{d_Y}(f(x), \epsilon)) \subset X \) es abierta.  
Además, \( x \in f^{-1}(B_{d_Y}(f(x), \epsilon)) \), ya que \( f(x) \in B_{d_Y}(f(x), \epsilon) \).

Entonces, como \( f^{-1}(B_{d_Y}(f(x), \epsilon)) \) es un abierto que contiene a \( x \), por definición de la topología métrica, existe \( \delta > 0 \) tal que:
\[
B_{d_X}(x, \delta) \subset f^{-1}(B_{d_Y}(f(x), \epsilon)).
\]
Aplicando \( f \)
\[
f(B_{d_X}(x, \delta)) \subset B_{d_Y}(f(x), \epsilon).
\]
lo cual es a lo que queremos llegar

\vspace{1em}
Supongamos que:
\[
\forall x \in X,\ \forall \epsilon > 0,\ \exists \delta > 0 : f(B_{d_X}(x, \delta)) \subset B_{d_Y}(f(x), \epsilon).
\]
Queremos probar que \( f \) es continua.  
Sea \( V \subset Y \) un abierto

P.D. \( f^{-1}(V) \) es abierto en \( X \).  

Sea \( x \in f^{-1}(V) \), entonces \( f(x) \in V \), y como \( V \) es abierto, existe \( \epsilon > 0 \) tal que:
\[
B_{d_Y}(f(x), \epsilon) \subset V.
\]
Por hipótesis, existe \( \delta > 0 \) tal que:
\[
f(B_{d_X}(x, \delta)) \subset B_{d_Y}(f(x), \epsilon) \subset V.
\]
Entonces:
\[
B_{d_X}(x, \delta) \subset f^{-1}(V).
\]
Por lo tanto, \( f^{-1}(V) \) es abierto, por lo que queda demostrado el ejercicio

\paragraph{12}
\textit{Demostrar que la métrica uniforme $\rho$ es métrica.}

Por definición, $\rho((x_n),(y_n))=\sup{\{\overline{d}(x_n,y_n)\}}$, donde $\overline{d}(x,y) \leq 1$ para cada $x,y \in \mathbb{R}$, por lo que $\rho$ está bien definida.
Es claro que $\rho((x_n),(x_n))=0$, además, $(x_n) \neq (y_n)$ implica que existe un natural $m$ con $\rho((x_n),(y_n))>=\overline{d}(x_m,y_m)>0$. Por tanto, $\rho((x_n),(y_n))=0$, si y sólo si $(x_n)=(y_n)$.
La simetria se hereda de la metrica acotada, $\rho((x_n),(y_n))=\sup{\{\overline{d}(x_n,y_n)\}}=\sup{\{\overline{d}(y_n,x_n)\}}=\rho((y_n),(x_n))$. Finalmente, veamos la desigualdad triangular.
\begin{align*}
    \rho((x_n),(y_n)) &\leq \sup{\{\overline{d}(x_n,z_n)+\overline{d}(z_n,y_n)\}} \\
                      &\leq \sup{\{\overline{d}(x_n,z_n)\}}+\sup{\{\overline{d}(z_n,y_n)\}}\\
                      &=\rho((x_n),(z_n))+\rho((z_n),(y_n)).
\end{align*}
\paragraph{13}
\textit{Sea $A=\{(x_n)_{n\in\mathbb{N}}\in\mathbb{R}^\omega : \exists N\in\mathbb{N}: x_n =0 ; n \geq N \}$
Hallar $\overline{A}$ en top. uniforme}

\paragraph{14}
\textit{Sea A del ejercicio anterior, hallar $\overline{A}$ en top. cajas}

Veamos que $A=\overline{A}$. Sea $(x_n) \notin A$, luego existe una sucesión extrictamente creciente de naturales $(n_k)$ tal que $x_{n_k} \neq 0$ para cada $k \in \mathbb{N}$. Como $\mathbb{R}$ es $T_1$, existen vecindades abiertas $U_{n_k}$ de $x_{n_k}$ que no contienen al $0$.
Tomemos $U=\prod_{k=1}^{\infty}U_{n_k} \times \prod_{n \neq n_k} \mathbb{R}$. Es facil ver que $(x_n) \in U$. Ahora sea $(y_n) \in A \cap U$, por definición existe $N$ tal que $n>N$ entonces $y_n=0$, pero al ser $n_k \to \infty$, para algun $k$ se cumple que
$n_k>N$ y $y_{n_k} \in U_{n_k}$, contradiciendo el hecho que $0 \notin U_{n_k}$. Por tanto $A \cap U = \emptyset$ y $A=\overline{A}$.
\paragraph{15}
\textit{Demostrar que $f^{-1}(Fr_Y(B))\subset Fr_X(f^{-1}(B))$}

La frontera se pueden escribir como:
\[
\mathrm{Fr}_Y(B) = \overline{B} \cap \overline{Y - B}
\quad \text{y} \quad
\mathrm{Fr}_X(f^{-1}(B)) = \overline{f^{-1}(B)} \cap \overline{X - f^{-1}(B)}
\]
Sea \( x \in f^{-1}(\mathrm{Fr}_Y(B)) \).  
Entonces:
\[
f(x) \in \mathrm{Fr}_Y(B) = \overline{B} \cap \overline{Y - B}
\Rightarrow
f(x) \in \overline{B} \quad \text{y} \quad f(x) \in \overline{Y - B}
\]
Aplicando la propiedad de clasura:
\[
x \in f^{-1}(\overline{B}) \subset \overline{f^{-1}(B)}
\quad \text{y} \quad
x \in f^{-1}(\overline{Y - B}) \subset \overline{f^{-1}(Y - B)}
\]
Pero como:
\[
f^{-1}(Y - B) = X - f^{-1}(B)
\Rightarrow
\overline{f^{-1}(Y - B)} = \overline{X - f^{-1}(B)}
\]
Entonces:
\[
x \in \overline{f^{-1}(B)} \cap \overline{X - f^{-1}(B)} = \mathrm{Fr}_X(f^{-1}(B))
\]
Por lo tanto:
\[
f^{-1}(\mathrm{Fr}_Y(B)) \subset \mathrm{Fr}_X(f^{-1}(B))
\]

\paragraph{16}
\textit{Sea $h:\mathbb{R}^\omega\rightarrow\mathbb{R}^\omega$ definida por: $h((x_n)_{n\in\mathbb{N}})=(a_n x_n +b_n)_{n\in\mathbb{N}}$. Ver si h es homeomorfismo en $\mathbb{R}^\omega$ bajo top. cajas}

Si $a_m=0$ para algun $m \in \mathbb{N}$, dado $(x_n)$, definimos $(y_n)$ por $y_n=x_n$ si $n \neq m$, $y_m=x_m+1$. Es claro que $(x_n) \neq (y_n)$, pero $h((x_n))=h((y_n))$, por lo que h no es biyectiva y por tanto, no puede ser homeomorfismo.
Supongamos entonces que $a_n \neq 0$ para cada $n \in \mathbb{N}$. Observe que $h^{-1}((x_n))=(\frac{x_n-b_n}{a_n})$ es la función inversa de $h$. Tanto $h$ como $h^{-1}$ son de la forma $f((x_n))=(c_n x_n+d_n)$, por lo que basta probar que esta función
es continua en la topología por cajas. Sea $p_n((x_n))=c_nx+d_n$, es facil ver que es continua para cada $n \in \mathbb{N}$, y sea $U=\prod_{i=1}^{m} U_{n_i} \times \prod_{n\neq n_i} \mathbb{R}$ un abierto en $\mathbb{R}^\omega$,  $f^{-1}(U)=\cap_{i=1}^{m} p^{-1}_{n_i}(U_{n_i})$ es una
intersección finita de abiertos, por lo que es abierta y $f$ es continua. Por tanto, $h$ es un homeomorfismo.
\end{document}
