\documentclass[12pt]{article}
\usepackage[spanish]{babel}
\usepackage{amssymb}
\usepackage{amsmath}
\usepackage{geometry}
 \geometry{
 letterpaper,
 total={170mm,237mm},
 left=20mm,
 top=15mm,
 }
\usepackage{setspace}
\spacing{1.5}
\setlength{\parindent}{0pt}
%Falta agregar una portada bien hecha aparte.
\LARGE{\title{Tareas de tercer parcial-Topología}}
\author{Alumnos: \\Arturo Rodriguez Contreras - 2132880 \\
Jonathan Raymundo Torres Cardenas - 1949731\\
Praxedis Jimenes Ruvalcaba \\
Erick Román Montemayor Treviño - 1957959 \\
Alexis Noe Mora Leyva \\
Everardo Flores Rivera - 2127301}
\begin{document}
\maketitle

\paragraph{1}
\textit{Sea Y subespacio de X con U, V separación en Y. Entonces $\overline{U}\cap V=\emptyset$ y $U\cap\overline{V}=\emptyset$}

\paragraph{2}
\textit{Sean X, Y esp. top. y $h:X\rightarrow Y$ un homeomorfismo. Demostrar que si C es componente de X, entonces h(C) es componente de Y}

\paragraph{3}
\textit{Sea $Y\subset X$ un subespacio de un esp. top. X. Y es compacto en X ssi toda cubierta abierta para Y por abiertos de X contiene una subcolección finita de abiertos en Y que lo cubren.}

$(\Rightarrow)$ Sea Y compacto y $A=\{A_\alpha\}_\alpha\in J$ una cubierta de Y de abiertos de X. Entonces $\{A_\alpha\cap Y|\alpha\in J\}$ tambien es una cubierta de Y por conjuntos abiertos en Y bajo la topologia del subespacio; como Y es compacto, existe una subcolección finita de dicha colección.

$(\Leftarrow)$ Ahora $A=\{A_\alpha\}$ es una cubierta de de Y de abiertos en X y por hipótesis existe una subcubierta finita $\{A_{\alpha_1},...,A_{\alpha_n}\}$. Entonces $\{A_{\alpha_1}\cap Y,...,A_{\alpha_n}\cap Y\}$ es recubrimiento finito de Y con abiertos en Y.


\paragraph{4}
\textit{La compacidad es una invariante topologica bajo continuidad.}

\paragraph{5}
\textit{Sea $f:X\times Y$ compacto y $T_2$. f es continuaen X ssi el conjunto $G_f =\{(x,f(x)):x\in X\}$ es cerrado.}

\paragraph{6}
\textit{Hallar un espacio 1-num pero no 2-num.}

Topología uniforme R omega
\paragraph{7}
\textit{Un subespacio de un espacio 2-num es 2-num. Hallar un contraejemplo del teorema de Lindelof.}

Sea X un es espacio 2-num y A subespacio de X. Entonces existe una base numerable $\mathfrak{B}$, entonces $\{B\cap A | B\in\mathfrak{B}\}$ es una base numerable para el subespacio A.
\paragraph{8}
\textit{El producto finito de Lindelof no es Lindelof}

\paragraph{9}
\textit{Hallar un $T_3$ que no es $T_4$}

\paragraph{10}
\textit{Sea $(X,\tau_X)$ un  espacio top. $T_1$. Demostrar que X es normal ssi para cada $A\subset X$ cerrado y U abierto en X tal que $A\subset U$, existe V abierto en X tal que $A\subset\overline{V}\subset U$}

\paragraph{11}
\textit{Si X es $T_2$ y compacto, entonces X es normal.}

\end{document}