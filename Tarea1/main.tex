\documentclass[12pt]{article}
\usepackage[spanish]{babel}
\usepackage{amssymb}
\usepackage{amsmath}
\usepackage{geometry}
 \geometry{
 letterpaper,
 total={170mm,237mm},
 left=20mm,
 top=15mm,
 }
\usepackage{setspace}
\spacing{1.5}
\setlength{\parindent}{0pt}
%Falta agregar una portada bien hecha aparte.
\LARGE{\title{Tareas de primer parcial-Topología}}
\author{Alumnos: \\Arturo Rodriguez Contreras - 2132880 \\
Jonathan Raymundo Torres Cardenas - 1949731\\
Praxedis Jimenes Ruvalcaba \\
Erick Román Montemayor Treviño - 1957959 \\
Alexis Noe Mora Leyva \\
Everardo Flores Rivera - 2127301}
\begin{document}
\maketitle

\paragraph{1}
\textit{¿Es la unión de topologías una topología?}

Sea $X = \{ a, b, c \}$, $\tau_{1} = \{\emptyset, X, \{a\} \}$, $ \tau_{2} = \{\emptyset, X, \{b\} \}$
se tiene que $\tau_{1}$ es topología ya que contiene al conjunto vacio, $X$, contiene las uniones arbitrarias
$\{a\} \cup X = X \in \tau_{1}$, y tambien
$\{a\} \cup \emptyset = \{a\} \in \tau_{1}$
y $\emptyset \cup X = X \in \tau_{1}$
y contiene a las intersecciones finitas de sus elementos, 
de igual forma se sigue que $\tau_{2}$ es topologia de $X$. La union de las dos topologías es
$U = \tau_{1} \cup \tau_{2} = \{ \emptyset, X, \{a\}, \{b\} \}$
lo cual no es topolog\'ia, ya que $\{a\} \cup \{b\} = \{a, b\} \notin U$, por lo tanto, no
necesariamente la unión de topologías es una topología.\\

\paragraph{2}
\textit{Demostrar que $\tau_{\mathbb{N}}$ es topologia.}

Se tiene por definicion que $\{\emptyset, X\} \subset \tau_{\mathbb{N}}$. Ahora, sea $\{U_{a}\}_{a \in J}$ una coleccion de elementos en $\tau_{\mathbb{N}}$,
y $U = \bigcup_{a \in J} U_{a}$. Queremos ver que $U \in \tau_{\mathbb{N}}$, para
esto observemos que $X - U = (\bigcup_{a \in J} U_{a})^{c} $ por leyes de De Morgan es igual a
$\bigcap_{a \in J} U_{a}^{c}$, sabemos por teorema que la intersecci\'on arbitraria de conjuntos contables es tambien
contable, entonces $\bigcap_{a \in J} U_{a}^{c} \in \tau_{\mathbb{N}}$. 

Luego, tomemos$\{U_{a}\}_{a \in J}$ una colecci\'on finita de elementos en
$\tau_{\mathbb{N}}$, y sea $U = \bigcap_{a \in J}U_{a}$ entonces tenemos $X - U = (\bigcap_{a \in J}U_{a})^{c}$ por leyes de DeMorgan es igual a $\bigcup_{a \in J}U_{a}^{c}$ y por teorema
la union finita de conjuntos contables es tambien
contable. Entonces $X - U$ es contable, por lo cual se tiene que $U \in \tau_{\mathbb{N}}$ entonces $\tau_{\mathbb{N}}$ esta
cerrado por intersecci\'on finita, como consequente es una topolog\'ia.\\

\paragraph{3}
\textit{Verificar si $\tau_{\infty}$ es topologia.} 

Sea $X=\mathbb{R}$, sea $U_1 = (-\infty,0),U_2 = (0,\infty)$, claramente $U_1,U_{2}\hspace{2px}\in\hspace{2px}\tau_{\mathbb{N}}$, pero $U=U_1\cup U_2 = (\infty,0)\cup(0,\infty)\notin \tau_{\infty}$
ya que $\mathbb{R}-U=\{0\}$ no es infinito. Por lo tanto no cumple el axioma de uniones arbitrarias de topología.\\ $\therefore\tau_{\infty}$ no es topología.\\

\paragraph{4}
\textit{Demostrar que $(0,1)=\bigcup\limits_{n\in\mathbb{N}-\{1\}}[\frac{1}{n},1)$.}

Veamos que $(0,1)\subset \bigcup\limits_{n\in\mathbb{N}-\{1\}}[\frac{1}{n},1)$.

Sea $x\in(0,1)$, esto es $0<x<1$ , por propiedad arquimediana existe $ N \in\mathbb{N}\hspace{4px}t.q \hspace{4px} \forall n\geq N \hspace{4px} 0< \frac{1}{n} \leqslant  x<1$ entonces $x\in[\frac{1}{n},1)$
entonces $x\in\bigcup\limits_{n\in\mathbb{N}-\{1\}}[\frac{1}{n},1)$\\Por tanto $(0,1)\subset\bigcup\limits_{n\in\mathbb{N}-\{1\}}[\frac{1}{n},1)$

Ahora veamos la otra contención.

Sea $x\in\bigcup\limits_{n\in\mathbb{N}-\{1\}}[\frac{1}{n},1)$, entonces $0<\frac{1}{n_0}\leq x<1$ para algun $n_0\in\mathbb{N}-\{1\}$\\$x\in(0,1)$\\ Por tanto $(0,1)\supset\bigcup\limits_{n\in\mathbb{N}-\{1\}}[\frac{1}{n},1)$

Esto demuestra la igualdad de los conjuntos.
\paragraph{5}
\textit{Verificar que $\beta_K$ satisface el teorema de creación de topologías.}

Demostraremos que para todo par $B_1$ y $B_2$ de basicos, existe un basico $B_3$ con $B_3\in B_1 \cap B_2$, esto lo dividiremos en tres casos.
\vspace{0.15cm}

Caso 1: $B_1=(a,b)$ y $B_2=(c,d)$.

Los básicos son iguales a los de la topología euclidiana, para la cual sabemos que cumple el teorema de creación de topologías.
\vspace{0.15cm}

Caso 2: $B_1=(a,b)$ y  $B_2=(c,d)-k$.

Notemos que $B_2=B_e-K$ con $B_e=(c,d)$, es decir un básico de la topología euclidiana.
Vemos que \(B_1\cap(B_e-k)=(B_1\cap B_e)-k\) y sabemos que si \(x\in B_1 \cap B_2\), entonces \(x\in B_1 \cap B_e\) y $x\notin K$. 
Pero como \(B_1\) y \(B_e\) son básicos de la topología euclidiana, existe un \(x\in B_{\alpha} \subset (B_1\cap B_e)\) tal que \(B_{\alpha}\) es basico de la topología euclidiana.
Luego $x \in B_\alpha -K \subset B_1 \cap B_2$.
\vspace{0.15cm}

Caso 3: $B_1=(a,b)-k$ y $B_2=(c,d)-k$.

Notemos que \(((a,b)-k)\cap((c,d)-k)=((a,b)\cap(c,d))-k\), por lo que obtendríamos un caso análogo al anterior. 

Como se cumple con el teorema de creación de topologías para todo caso,  $\beta_K$ es una topología.
\paragraph{6}
\textit{Verificar que $B_S=\{(a,b]:a<b$\} es base de algunas topologias}

Sean $B_1=(a,b], B_2=(c,d] \in B_S$, observemos que $B_1 \cap B_2 = (min\{a,c\},max\{b,d\}]$
Como $B_1 \cap B_2 \in B_S$, se concluye por el teorema de creación de topologías que $B_S$ es base de algunas topologias.


\paragraph{7}
\textit{Verificar las comparaciones entre $\tau_S, \tau_L$ y entre $\tau_S, \tau_K$}
\\Consideraremos $B_S=\{(a,b] : a<b\}$ como la base la de la topología del límite superior ($\tau_S$), $B_L=\{[a,b) : a<b\}$ como la base de la topología del límite inferior ($\tau_L$), $B_k=\{(a,b) : a<b\}\bigcup\{(a,b)-k : a<b, k=\frac{1}{n}, n\in\mathbb{N}\}$\\
\\1. Verificamos las comparaciones entre $\tau_S$ y $\tau_L$
\\Tomemos $(0,1] \in B_S $ y $x=1$ tal que $x \in (0,1]$. Considere $[a,b) \in B_L$ tal que $x\in[a,b)$. Entonces $a\leq x<b$ pero como $x=1<b$ entonces $b \notin (0,1]$, $[a,b)\not\subset(0,1]$
\\\(\therefore\)Por teorema de comparación $\tau_S \not\subset \tau_L$
\\Tomemos $[0,1) \in B_L $ y $x=0$ tal que $x \in [0,1)$. Considere $(a,b] \in B_S$ tal que $x\in(a,b]$. Entonces $a<x\leq b$ pero como $a<x=0$ entonces $a \notin [0,1)$, $(a,b]\not\subset[0,1)$
\\\(\therefore\)Por teorema de comparación $\tau_L \not\subset \tau_S$
\\\(\therefore\)$\tau_S$ y $\tau_L$ no son comparables.\\
\\2. Verificamos ahora la comparación entre $\tau_S$ y $\tau_K$
\\Sea $(0,1] \in B_S$ y $x=1$ tal que $x\in(0,1]$. Considere $(a,b)\in B_K$ tal que $x \in (a,b)\iff a<1<b$, entonces $b\notin(0,1]$, $(a,b)\not\subset(0,1]$
\\\(\therefore\)Por teorema de comparación $\tau_S \not\subset \tau_K$
\\Caso 1: básicos de la forma $(a,b)$
Sea $(a,b)\in B_k$ y $x\in(a,b)$. Considere $x \in (a,x]\in B_S$. Sean $y\in(a,x] \iff a<y \leq x$ y como $x\in(a,b)\iff a<x<b$, $a<y<b$, es decir $y\in(a,b), y\in(a,x]\subset(a,b)$
\\\(\therefore\)Por teorema de comparación se cumple para el primer caso
\\Caso 2: Básicos de la forma $(a,b)-k$
\\Sea $(a,b)-l \in B_K$ y $x\in(a,b)-k$. Considere $a,x]\in B_S$, si $a<\frac{1}{n}$ para algún $n \in \mathbb{N}$ entonces existe $q_i\in k,$ tal que $a<q_i<x, i\in\{0,1,2,...\}$, entonces tomamos $q=max\{q_0,q_1,q_2,...\}$, así describiendo $(q,x]\in B_S$, y como $a<q<x<b$, entonces, $(q,x]\subset(a,b)-k$
\\\(\therefore\)Por teorema de comparación se cumple para este segundo caso también, por lo que $\tau_K \subset \tau_S$
\\\(\therefore\)La topología del límite superior es más fina que la k-topología.
\paragraph{8}
\textit{Demostrar si $\tau_{\mathbb{R}^2}=\tau_{\mathbb{R} \times \mathbb{R}}$}
\\Primero vemos que \(\tau_{\mathbb{R} \times \mathbb{R}}\)  es más fina que \( \tau_{\mathbb{R}^2}\).
Sea \(y\in(x,\epsilon)\) donde x denota el centro y el espilón el radio de la bola. Este es un básico en la topología \( \tau_{\mathbb{R}^2}\) y al ser topología, podemos hallar un básico \((y,\delta)\subset(x,\epsilon)\) con claramente \(\delta<\epsilon\).
\\Si quisieramos inscribir un rectángulo, consideraríamos a \(\delta\) como la diagonal mayor, entonces, \(\delta=\frac{\sqrt{a^2+a^2}}{2}\), siendo a los lados de nuestro cuadrado. Despejando obtendríamos la expresión \(\sqrt{2}\delta=a\). Considerando a \(a<\sqrt{2}\delta\), obtendríamos el cuadrado \((y-a,y+a)\)x\((y-a,y+a)\subset(y,\delta)\subset(x,\epsilon)\). por transtitivadad de contenciones obtendríamos:
\\\(y\in(y-a,y+a)\)x\((y-a,y+a)\subset(x,\epsilon)\)
\\\(\therefore\tau_{\mathbb{R}^2}\subset\tau_{\mathbb{R} \times \mathbb{R}}\)
\\Ahora verificamos la otra contención.
\\Sea \(x\in(a,b)\)x\((c,d): a<b y c<d\),
\\tomamos \(\delta=min\{(x,(a,b)),(x,(c,d)),(x,(a,c)),(x,(b,d))\}\), entonces, La bola \((x,\frac{\delta}{2})\subset(a,b)\)x\((c,d)\), donde \(x\in(x,\frac{\delta}{2})\)
\\\(\therefore\tau_{\mathbb{R} \times \mathbb{R}}\subset\tau_{\mathbb{R}^2}\)
\\\(\therefore\) como se cumplen ambas contenciones; \(\tau_{\mathbb{R} \times \mathbb{R}}=\tau_{\mathbb{R}^2}\)

\paragraph{9}
\textit{Terminar paso inductivo del teorema de las proyecciones}

\paragraph{10}
\textit{Demuestra que $(A \times B) \cap (C \times D)= (A \cap C) \times (B \cap D)$.}

Veamos que $(A \times B) \cap (C \times D)  \subset (A \cap C) \times (B \cap D)$.

Sea $(x,y) \in (A \times B) \cap (C \times D)$. Por definición de intersección $(x,y)\in A\times B$ y $(x,y) \in C\times D$. Además, por definición
de producto cruz $x \in A $ y $y \in B$, $x \in C$ y $y \in D$. Reescribiendo obtenemos $x\in A$ y $x\in C$, $y \in B$ y $y\in D$ \textit{i.e.}  $(x,y) \in (A \cap C)\times (B \cap D)$.

Ahora veamos la otra contención.

De forma análoga, sea $(x,y) \in (A \cap C) \times (B \times D)$ luego  $x \in A $ y $y \in B$, $x \in C$ y $y \in D$ y $(x,y) \in (A \times B) \cap (C \times D)$.

Esto demuestra la igualdad de los conjuntos.
\paragraph{11}
\textit{Verificar que la topologia del orden $\tau(\beta_O)$ es topología.}

\paragraph{12}
\textit{Verificar si en $\mathbb{N}$,  $\tau_O =\tau_d$.}

Veamos que las topologías coinciden.

Sea $U \in \tau_O$, observe que $U= \cup_{x\in U} \{x\}$ es una unión de básicos de $\tau_d$ \textit{i.e.} $\tau_O \subset \tau_d$.

De forma análoga, sea $U \in \tau_d$, y $I_x=(x-1,x+1)$ si $x \neq 1$, $I_1=[1,2)$. Note que $U=\cup_{x\in U} I_x$ es una unión de 
basicos de $\tau_O$ y por tanto $\tau_d \subset \tau_O$. Esto demuestra que, en $\mathbb{N}$, $\tau_O =\tau_d$.

\paragraph{13}
\textit{Verificar que el orden lexicografico genera un orden en $\mathbb{R}$.}

\paragraph{14}
\textit{Verificar las comparaciones entre $\tau_o(\mathbb{R}^2)$ y $\tau_{d\times \mathbb{R}}$}

Primero demostraremos que $\tau_o(\mathbb{R}^2)\subset \tau_{d\times \mathbb{R}}$

Sea un $B=((a,b),(c,d))\in B_o$ y $x=(u,v)\in B$

Caso trivial: si $a=c$ 

$x=(a,v)$ donde $b<v<d$, entonces $x\in \{a\times (b,d)\}\in B_{d\times \mathbb{R}}\subset B$

Caso 2: si $a\neq b$

Caso 2.1: si 
$x=(a,v)$, entonces $v\in(b,\infty)$ Sea $B'=\{a\}\times (b,\infty)\in B_{d\times \mathbb{R}}$
entonces $x\in B'\subset B$

Caso 2.2: $x=(c,v)$, entonces $v\in(-\infty,d)$ Sea $B'=\{c\}\times (-\infty,d)\in B_{d\times \mathbb{R}}$
entonces $x\in B'\subset B$

Caso 2.3: $x=(u,v)$, con $a<u<c$, entonces $v\in\mathbb{R}$ Sea $B'=\{u\}\times\mathbb{R}\in B_{d\times \mathbb{R}}$
entonces $x\in B'\subset B$

Por lo tanto $\tau_o(\mathbb{R}^2)\subset \tau_{d\times \mathbb{R}}$

Ahora demostraremos que $\tau_o(\mathbb{R}^2)\supset  \tau_{d\times \mathbb{R}}$

Sea $B=\{\{x\}\times(a,b)\}\in\tau_{d\times \mathbb{R}}$ tal que $(x,y)\in B$ esto es que $a<y<b$

Sea $B'=((x,a),(x,b))\in B_o$ observemos que $(x,y)\in B'$ y ademas $B'\subset B$ 

Por lo tanto $\tau_o(\mathbb{R}^2)=  \tau_{d\times \mathbb{R}}$

\paragraph{15}
\textit{Demuestre que $\overline{A \cap B} \subset \overline{A} \cap \overline{B}$.}

Sabemos que $A \subset \overline{A}$ y $B \subset \overline{B}$. Luego $A \cap B \subset \overline{A} \cap \overline{B}$. La cerradura de un conjunto es siempre
cerrado y la intersección de cerrados es cerrada, por lo que $A \cap B$ está contenido en un cerrado y la cerradura es el cerrado más pequeño que contiene al conjunto. Por lo tanto
$\overline{A \cap B} \subset \overline{A} \cap \overline{B}$.

\paragraph{16}
\textit{Verificar si $\overline{\bigcup\limits_{\alpha\in J}A_{\alpha}}=\bigcup\limits_{\alpha\in J}\overline{A_{\alpha}}$}

El resultado es en general falso. Sea $X=\mathbb{R}$ y $A_n=[1/n,1)$.
Luego $[0,1]=\overline{\bigcup\limits_{\alpha\in J}A_{\alpha}} \neq \bigcup\limits_{\alpha\in J}\overline{A_{\alpha}} =(0,1]$.
\paragraph{17}
\textit{Verificar si $X-\overline{A}=\overline{X-A}$}

El resultado es en general, falso. Sea $X=\mathbb{R}$ y $A=\{0\}$ bajo la topología usual.
 Luego $\mathbb{R}-\{0\} = X-\overline{A}\neq \overline{X-A} = \mathbb{R}$

\paragraph{18}
\textit{Considere $([0,1])^2$ bajo $\tau_{\mathbb{R}_L\times\mathbb{R}_S}$ hallar $Int([0,1]^2)$}

Podemos deducir que los b\'asicos de $\tau_{\mathbb{R}_L\times\mathbb{R}_S}$
son de la forma $[a, b) \times (c, d]$, por lo cual el abierto m\'as
grande que est\'a en $Int([0,1]^2)$ es $[0, 1) \times (0, 1]$

\paragraph{19}
\textit{Verificar si $Int(A\cap B)=Int(A)\cap Int(B)$}

Es un hecho conocido que $\text{Int}(A)=\overline{A^c}^c$ y que $\overline{A \cup B}=\overline{A} \overline{B}$. Aplicando estas propiedades tenemos que 
\begin{align*}
    \text{Int}(A \cap B) &=\overline{(A \cap B)^c}^c \\
                         &=(\overline{A^c \cup B^c})^c \\
                         &=(\overline{A^c} \cup \overline{B^c})^c \\
                         &=\overline{A^c}^c \cap \overline{B^c}^c \\
                         &=\text{Int}(A) \cap \text{Int}(B).
\end{align*}


\paragraph{20}
\textit{Verificar que si $D_{1}, D_{2}$ son densos y abiertos en $X$, entonces $D_{1} \cap D_{2}$ es denso en $X$}

Sea U un abierto arbitrario de X, tenemos que por
asociatividad de la intersecci\'on
$U \cap (D_{1} \cap D_{2}) = (U \cap D_{1}) \cap D_{2}$,
y $(U \cap D_{1})$ es abierto por la segunda axioma de
topolog\'ia. Ahora bien, tenemos que la intersecci\'on de
cualquier abierto con $D_{2}$ es no vacio ya que $D_{2}$ es
denso, entonces $(U \cap D_{1}) \cap D_{2} \neq \varnothing$.
Juntando todo lo que tenemos, $U \cap (D_{1} \cap D_{2})$
$= (U \cap D_{1}) \cap D_{2} \neq \varnothing$, por teorema
se tiene que $D_{1} \cap D_{2}$ es denso.

\paragraph{21}
\textit{Verificar la convergencia de $\left\{\frac{1}{n}\right\}$ donde $n \in N $, bajo la topolog\'ia del l\'imite superior }

Sea la sucesi\'on \(\left\{\frac{1}{n}\right\}\), donde \(n\) pertenece a los naturales. Bajo la topolog\'ia del l\'imite superior, la convergencia se verifica observando que la sucesi\'on tiende a 0 cuando \(n \to \infty\). Para cualquier \(\epsilon > 0\), existe un \(N\) tal que para todo \(n > N\), \(\frac{1}{n} < \epsilon\), lo cual demuestra que la sucesión converge a 0 bajo esta topolog\'ia. 

Por lo tanto \(\left\{\frac{1}{n}\right\}\) converge a 0 bajo la topolog\'ia del l\'imite superior

\paragraph{22}
\textit{Verificar si \(\tau_{\infty}\) es Hausdorff.}

Sean \(x, y \in X\) dos puntos distintos, consideremos los conjuntos abiertos \(U = X - \{y\}\) y \(V = X - \{x\}\), abiertos en \(\tau_{\infty}\)
vemos \(U \cap V = X - \{x, y\}\), que no es vac\'io

Se obiene que \(x \in U\) y \(x \in U\), esto implica que la intersecci\'on \(U\) y \(V\) no es vacía, lo que significa que no podemos separar los puntos
\(x\) y \(y\) por conjuntos abiertos disjuntos, por lo tanto \(\tau_{\infty}\) no es Hausdorff.

\paragraph{23}
\textit{Demostrar que \(\overline{A \times B} = \overline{A} \times \overline{B}\).}

Demostramos que $\overline{A \times B} \subseteq \overline{A} \times \overline{B}$. Sea $(x, y) \in \overline{A \times B}$. Esto significa que para toda vecindad $U \times V$ de $(x, y)$, se tiene que:
\[
(U \times V) \cap (A \times B) \neq \emptyset.
\]
Dado que la intersección del producto es el producto de las intersecciones:
\[
(U \times V) \cap (A \times B) = (U \cap A) \times (V \cap B) \neq \emptyset,
\]
se sigue que $U \cap A \neq \emptyset$ y $V \cap B \neq \emptyset$.
Por lo tanto, $x \in \overline{A}$ y $y \in \overline{B}$, lo que implica que $(x, y) \in \overline{A} \times \overline{B}$. Así, obtenemos la inclusión deseada.

Demostramos que $\overline{A} \times \overline{B} \subseteq \overline{A \times B}$

Sea $(x, y) \in \overline{A} \times \overline{B}$, es decir, $x \in \overline{A}$ y $y \in \overline{B}$. Esto significa que:
\[
\forall \text{ vecindad } U \text{ de } x, \quad U \cap A \neq \emptyset,
\]
\[
\forall \text{ vecindad } V \text{ de } y, \quad V \cap B \neq \emptyset.
\]
Tomando cualquier vecindad $U \times V$ de $(x, y)$, se tiene que:
\[
(U \cap A) \times (V \cap B) \neq \emptyset.
\]
Es decir,
\[
(U \times V) \cap (A \times B) \neq \emptyset.
\]
Dado que esto es cierto para toda vecindad $(U \times V)$ de $(x, y)$, se concluye que $(x, y) \in \overline{A \times B}$. Así, obtenemos la segunda inclusión.

Como hemos demostrado ambas inclusiones, concluimos que:
\[
\overline{A \times B} = \overline{A} \times \overline{B}.
\]

\paragraph{24}
\textit{Demostrar que \(\tau_Y\) es la topología del subespacio de \( X \).}

1. \( \emptyset \) y \( Y \) est\'an en \( \tau_Y \) se sabe que \( \emptyset \in \tau_X \) y \( X \in \tau_X \)
entonces \( \emptyset \cap Y = \emptyset \) y \( X \cap Y = Y \) están en \( \tau_Y \), por lo tanto \( \emptyset, Y \in \tau_Y \).

2. Cerradura de uniones: Sea \( \{ U_i \}_{i \in I} \) una colecci\'on de conjuntos de \( \tau_Y \)
se quiere demostrar que la unión de la colecci\'on \( \bigcup_{i \in I} U_i \), está en \( \tau_Y \), se tiene que

\[
\bigcup_{i \in I} U_i = \bigcup_{i \in I} (V_i \cap Y) = \left( \bigcup_{i \in I} V_i \right) \cap Y.
\]
entonces la unión de cualquier colecci\'on de conjuntos de \( \tau_Y \) pertenece a \( \tau_Y \).

3. Cerradura bajo intersecciones: sea \( \{ U_1, U_2, \dots, U_n \} \) ,es decir, \( U_i = V_i \cap Y \) para algún \( V_i \in \tau_X \) y para cada \( i \). 
Por conjuntos se tiene que
\[
\bigcap_{i=1}^n U_i = \bigcap_{i=1}^n (V_i \cap Y) = \left( \bigcap_{i=1}^n V_i \right) \cap Y.
\]

por lo tanto, la intersección de cualquier colecci\'on finita de conjuntos de \( \tau_Y \) pertenece a \( \tau_Y \), 
concluimos que \( \tau_Y \) es una topolog\'ia sobre \( Y \).

\paragraph{25}
\textit{Demostrar que \( U \) es abierto en \( X \) si y solo si\[
\overline{U \cap \overline{A}} = \overline{U \cap A}.
\] para toda \( A \subseteq X \).}

\end{document}